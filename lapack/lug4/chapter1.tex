\chapter{Essentials}\label{chapessentials}
\pagestyle{headings}

\section{LAPACK}

LAPACK is a library of Fortran~90 subroutines for solving the most commonly
occurring problems in numerical linear algebra\index{Fortran~90 subroutines}
\index{numerical linear algebra}.
It has been designed to be efficient on a wide range of modern
high-performance computers. The name LAPACK is an acronym for Linear
Algebra PACKage. 
\begin{center}
\begin{quote}
{\tt http://www.netlib.org/lapack/}
\end{quote}
\end{center}
A list of LAPACK Frequently Asked Questions (FAQ)\index{FAQ!LAPACK}
is maintained on this webpage.

LAPACK routines can be called by programs written in either Fortran~90 or C.
For C programs a file of C interfaces, {\tt lapack.e}, must be called; this is described in 
Chapter~\ref{chapcinterface}\index{C interface}.

\section{Problems that LAPACK can Solve}

LAPACK can solve systems of linear equations, linear least squares
problems (LSE), symmetric eigenvalue problems (SEP),
nonsymmetric eigenvalue problems (NEP), singular value problems (SVD),
generalized problems of the same kind (GLSE, GSEP, GNEP, GSVD), and CS decomposition.
LAPACK can also handle many
associated computations such as matrix factorizations
or estimating condition numbers.

LAPACK contains {\bf driver routines} for solving standard types of
problems\index{driver routines},
{\bf computational routines} to perform a distinct
computational task\index{computational routines}, and {\bf auxiliary routines} to perform a certain
subtask or common low-level computation.  Each driver routine
typically calls a sequence of computational routines. Taken as a whole, the computational routines
can perform a wider range of tasks than are covered by the driver routines.
Many of the auxiliary routines may be of use to numerical analysts
or software developers, so we have documented the Fortran source for
these routines with the same level of detail used for the LAPACK
routines and driver routines.

Dense and band matrices are provided for, but not general sparse matrices. 
In all areas, similar functionality is provided for real and complex matrices.
See Chapter~\ref{chapcontents} for a complete summary of the contents.

\section{Computers for which LAPACK is Suitable}

LAPACK is designed to give high efficiency\index{performance} on vector processors,
high-performance ``super-scalar'' workstations, and shared memory multiprocessors.
It can also be used satisfactorily on all types of scalar machines (PC's, workstations, mainframes).

\section{LAPACK Compared with LINPACK and EISPACK}

LAPACK has been designed to supersede LINPACK~\cite{dongarra79}\index{LINPACK}
and EISPACK~\cite{Smith76,Garbow77}\index{EISPACK}, principally by restructuring the 
software to achieve much greater efficiency, where possible, on modern
high-performance computers; also by adding extra functionality,
by using some new or improved algorithms, and by integrating the two sets
of algorithms into a unified package.

\section{LAPACK and the BLAS}

LAPACK routines are written so that as much as possible of the
computation is performed by calls to the
Basic Linear Algebra Subprograms (BLAS)~\cite{blas1,blas2,blas3}\index{BLAS}.
Highly efficient machine-specific implementations of the BLAS are
available for many modern high-performance computers. The BLAS
enable LAPACK routines to achieve high performance with portable code.
The methodology for constructing LAPACK routines in terms of calls to the BLAS
is described in Chapter~\ref{chapperformance}. 

The BLAS are not strictly speaking part of LAPACK, but Fortran~90 code
for the BLAS is distributed with LAPACK, or can be obtained separately
from {\em netlib}.  This code constitutes the ``model implementation''~\cite{blas2alg,blas3alg}.
\begin{quote}
{\tt http://www.netlib.org/blas/blas.tgz}
\end{quote}

The model implementation is not expected to perform as well as a specially tuned implementation
on most high-performance computers --- on some machines it may give {\it much}
worse performance ---  but it allows users to run LAPACK codes on machines that do not offer
any other implementation of the BLAS.

For information on available optimized BLAS libraries, as well as
other BLAS-related questions, please refer to the BLAS FAQ\index{FAQ!BLAS}:
\begin{quote}
{\tt http://www.netlib.org/blas/faq.html}
\end{quote}

\section{Availability of LAPACK}

The complete LAPACK package or individual routines from LAPACK\index{source code} are
freely available on {\em netlib}~\cite{Dongarra87e}\index{netlib} and can be obtained
via the World Wide Web or anonymous ftp.

The LAPACK homepage can be accessed on the World Wide Web via the URL address:
\begin{quote}
{\tt http://www.netlib.org/lapack/}
\end{quote}

Prebuilt LAPACK libraries\index{prebuilt libraries!lapack}\index{LAPACK!prebuilt libraries, availability of} are available on {\em netlib} for a variety of architectures:
\begin{quote}
{\tt http://www.netlib.org/lapack/archives/}
\end{quote}

The main {\em netlib} servers are:
\begin{center}
\begin{tabular}{l l}
Tennessee, U.S.A.  & {\tt http://www.netlib.org/} \\
New Jersey, U.S.A. & {\tt http://netlib.bell-labs.com/} \\
Kent, UK  & {\tt http://www.mirror.ac.uk/sites/netlib.bell-labs.com/netlib/master/} \\
Bergen, Norway     & {\tt http://www.netlib.no/}
\end{tabular}
\end{center}
Each of these sites is the master location for some significant part
of the netlib collection of software; a distributed replication facility
keeps them synchronized nightly.

There are also a number of mirror repositories\index{netlib!mirror
repositories} located around the world, and a list of these sites is maintained on {\em netlib}:
\begin{quote}
{\tt http://www.netlib.org/bib/mirrors.html}
\end{quote}

Most of the sites provide both ftp and http access.  If ftp and http
are difficult, a user may wish to send e-mail.  E.g.,
\begin{quote}
{\tt echo ``send index from lapack'' | mail netlib@www.netlib.org}
\end{quote}

General information about LAPACK can be obtained by contacting any of
the URLs listed above.  If additional information is desired, feel free
to contact the authors at {\tt lapack@cs.utk.edu}.

The complete package, including test code, in four
different Fortran data types, constitutes some 805,000 lines of Fortran
source and comments.

\section{Commercial Use of LAPACK}\index{commercial use}\index{LAPACK!commercial use of}

LAPACK is a freely available software package provided on the World Wide Web
via {\em netlib,} anonymous ftp, and http access.  Thus it can be included in commercial
packages (and has been).  We ask only that proper credit be given to
the authors by citing this users' guide as the official reference for LAPACK.

Like all software, this package is copyrighted.  It is not trademarked; however,
if modifications are made that affect the interface, functionality, or accuracy of the resulting 
software, the name of the routine should be changed.  Any modification to our software
should be noted in the modifier's documentation.

We will gladly answer questions regarding our software.
If modifications are made to the software, however, it is the
responsibility of the individuals/company who modified the routine to provide support.

\section{Installation of LAPACK}\index{installation}

To ease the installation process, prebuilt LAPACK libraries
are available on {\em netlib} for a variety of architectures:
\begin{quote}
{\tt http://www.netlib.org/lapack/archives/}
\end{quote}
Included with each prebuilt library archive is the make include
file {\tt make.inc} detailing the compiler options, and so on, used
to compile the library.
If a prebuilt library is not available for the specific architecture,
the user will need to download the source code from {\em netlib}:
\begin{quote}
{\tt http://www.netlib.org/lapack/lapack.tgz} \\
{\tt http://www.netlib.org/lapack/lapack-pc.zip}
\end{quote}
and build the library as instructed in the LAPACK Installation
Guide~\cite{lawn41,lawn81}.  Note that separate distribution tar/zip files
are provided for Unix/Linux and Windows 98/NT installations.
Sample {\tt make.inc} files for various
architectures are included in the distribution tar/zip file and will
require only limited modifications to customize for a specific
architecture.

Machine-specific installation hints are contained in the {\tt release\_notes}
file on {\em netlib}:
\begin{quote}
{\tt http://www.netlib.org/lapack/release\_notes}
\end{quote}

A comprehensive test suite\index{test suites}\index{reliability, see test suites} for the BLAS is provided in the LAPACK distribution and on
{\em netlib}, and it is highly recommended that this test suite, as well
as the LAPACK test suite, be run to ensure proper installation of the package.

Two installations guides are available for LAPACK.
A Quick Installation Guide (LAPACK Working Note 81)~\cite{lawn81} is
distributed with the complete package.  This Quick Installation Guide
provides installation instructions for Unix/Linux systems.  A comprehensive
Installation Guide (LAPACK Working Note 41)~\cite{lawn41}, which contains
descriptions of the testing and timings programs, as well as detailed non-Unix
installation instructions, is also available.
See also Chapter~\ref{chapinstall}.

\section{Documentation for LAPACK}\label{documentation}

This {\bf Users' Guide} gives an informal introduction to
the design of the package, and a detailed description of its contents.
Chapter~\ref{chapdocsoftware} explains the conventions used in the
software and documentation.

On-line manpages (troff files) for LAPACK routines, as well as for most of
the BLAS routines, are available on netlib.  These files are
automatically generated at the time of each release.  For more
information, see the {\tt manpages.tgz} entry on the {\tt lapack}
index on netlib.

\section{Support for LAPACK}\index{support}

LAPACK has been thoroughly tested before release, on many different
types of computers.
The LAPACK project supports the package in the sense that reports
of errors or poor performance will gain immediate attention from the
developers\index{bug reports}. 
Such reports --- and also descriptions of interesting 
applications and other comments --- should be sent to:
\begin{quote}
LAPACK Project\\
c/o J. J. Dongarra\\
Computer Science Department\\
University of Tennessee\\   
Knoxville, TN 37996-1301\\
USA\\   
Email: lapack@cs.utk.edu\\   
\end{quote}

\section{Errata in LAPACK}

A list of known problems, bugs, and compiler errors\index{errata} for LAPACK,
as well as an errata list for this guide, is maintained on {\em netlib}:
\begin{quote}
{\tt http://www.netlib.org/lapack/release\_notes}
\end{quote}

This errata file, as well as an FAQ (Frequently Asked Questions) file,
can be accessed via the LAPACK homepage.

\section{Other Related Software}\label{relsoftware}

As previously mentioned in the Preface, many LAPACK-related software
projects are currently available on netlib.

The ScaLAPACK (or Scalable LAPACK)\index{ScaLAPACK} library includes
a subset of LAPACK routines redesigned for distributed memory
message-passing MIMD computers and networks of workstations supporting
PVM and/or MPI.  It is written in a Single-Program-Multiple-Data (SPMD)
style using explicit message passing for interprocessor communication.
The goal is to have ScaLAPACK routines resemble their LAPACK
equivalents as much as possible.  Just as LAPACK
is built on top of the BLAS, ScaLAPACK relies on the PBLAS 
(Parallel Basic Linear Algebra Subprograms)\index{PBLAS} and the
BLACS (Basic Linear Algebra Communication Subprograms).
The PBLAS perform computations analogous to the BLAS but on matrices
distributed across multiple processors.
The PBLAS rely on the communication protocols of the BLACS.
The BLACS are designed for linear
algebra applications and provide portable communication across a wide variety
of distributed-memory architectures.  ScaLAPACK is portable on any
computer that supports MPI or PVM.  Clusters of computers that have MPI
or PVM can use the ScaLAPACK software.  In addition, a BLACS interface exists
for the Intel series (NX), and IBM SP series (MPL).

For more detailed information please refer to the ScaLAPACK Users' Guide~\cite{slug} or the ScaLAPACK homepage:
\begin{quote}
{\tt http://www.netlib.org/scalapack/}
\end{quote}
All questions/comments can be directed to {\tt scalapack@cs.utk.edu}.

Alternative language interfaces to earlier versions of LAPACK (or translations/conversions
of LAPACK) are available in Fortran~90 and Java.
\begin{quote}
{\tt http://www.netlib.org/lapack90/} \\ \index{LAPACK90}
{\tt http://www.netlib.org/java/f2j/} \index{JLAPACK}
\end{quote}
