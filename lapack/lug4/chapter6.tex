\chapter{Installing LAPACK Routines}\label{chapinstall}

\section{Points to Note}\label{chapinstallsec1}

For anyone who obtains the complete LAPACK package from {\em netlib}
(see Chapter~\ref{chapessentials}), a
comprehensive installation\index{installation!LAPACK} 
guide\index{installation guide} is provided. We recommend
installation of the complete package as the most convenient and reliable
way to make LAPACK available.

People who obtain copies of a few LAPACK routines from {\em netlib} need to be
aware of the following points:

\begin{enumerate}

\item 
For optimal performance, a small set of tuning parameters must be set
for each machine, or even for each configuration of a given machine
(for example, different parameters may be optimal for different numbers
of processors).
These values\index{machine parameters}, such as the block size, minimum block size, crossover
point below which an unblocked routine should be used, and others,
are set by calls to an inquiry function ILAENV.
The default version of ILAENV\indexR{ILAENV} provided with LAPACK uses generic values
which often give satisfactory performance, but
users who are particularly interested in performance may wish to
modify this subprogram or substitute their own version.
Further details on setting ILAENV for a particular environment
are provided in section~\ref{secilaenv}.

\item 
xLAMCH\index{installation!xLAMCH}\indexR{SLAMCH}
determines properties of the 
floating-point arithmetic at run-time, such as the machine epsilon,
underflow threshold,
overflow threshold, and related parameters. It works satisfactorily on
all commercially important machines of which we are aware, but will necessarily
be updated from time to time as new machines and compilers are produced.

\end{enumerate}

\section{Installing ILAENV}\label{secilaenv}

Machine-dependent\index{installation!ILAENV}\indexR{ILAENV}
parameters\index{machine parameters} such as the block size are set
by calls to an inquiry function which may be set with different
values on each machine.
The declaration of the environment inquiry function is:
\begin{verbatim}
      INTEGER FUNCTION ILAENV( ISPEC, NAME, OPTS, N1, N2, N3, N4 )
\end{verbatim}
where ISPEC, N1, N2, N3, and N4 are integer variables and
NAME and OPTS are CHARACTER*(*).  NAME specifies the subroutine name;
OPTS is a character string of options to the subroutine; and N1, N2, N3 and N4 are
the problem dimensions.
ISPEC specifies the parameter to be returned; the following values are
currently used in LAPACK:

\begin{verbatim}
*          ISPEC is INTEGER
*          Specifies the parameter to be returned as the value of
*          ILAENV.
*          = 1: the optimal blocksize; if this value is 1, an unblocked
*               algorithm will give the best performance.
*          = 2: the minimum block size for which the block routine
*               should be used; if the usable block size is less than
*               this value, an unblocked routine should be used.
*          = 3: the crossover point (in a block routine, for N less
*               than this value, an unblocked routine should be used)
*          = 5: the minimum column dimension for blocking to be used;
*               rectangular blocks must have dimension at least k by m,
*               where k is given by ILAENV(2,...) and m by ILAENV(5,...)
*          = 6: the crossover point for the SVD (when reducing an m by n
*               matrix to bidiagonal form, if max(m,n)/min(m,n) exceeds
*               this value, a QR factorization is used first to reduce
*               the matrix to a triangular form.)
*          = 9: maximum size of the subproblems at the bottom of the
*               computation tree in the divide-and-conquer algorithm
*               (used by xBDSDC, xGELSD and xSTEDC)
*          =10: IEEE NaN and Infinity arithmetic can be trusted not to trap
*          12 <= ISPEC <= 16: used by xHSEQR or one of its subroutines,
*               see IPARMQ for detailed explanation
\end{verbatim}

The three block size parameters, NB (ISPEC = 1), NBMIN (ISPEC = 2) and NX (ISPEC = 3), 
are used in many different\index{tuning!block size: NB, NBMIN, and NX}
subroutines (see Table~\ref{nbnx.tab}).
The parameter ISPEC = 5 is used in the same way as NBMIN (ISPEC = 2) by xTGSYL for a
different dimension.
This table only lists those routines which take direct action depending on the value of the
parameters; other routines may call ILAENV, but only to determine the size of work arrays,
and set the value of NAME (parameter number 2) to one of the routines in the table.
For example, SORGHR calls ILAENV with 'SORQRF' as parameter,
to determine the size of its work array,
and then calls SORQRF; SORQRF again calls ILAENV, this time with 'SORQRF' as parameter,
to determine the course of its computation.

ISPEC= 6 is used\index{tuning!SVD} in the driver 
routines xGELSD, xGELSS and xGESVD.
\indexR{SGELSD}\indexR{CGELSD}
\indexR{SGELSS}\indexR{CGELSS}
\indexR{SGESVD}\indexR{CGESVD}
ISPEC = 9 is used\index{tuning!divide and conquer} in the
divide and conquer routines xBDSDC, xGELSD and xSTEDC.
\indexR{SBDSDC}\indexR{CBDSDC}
\indexR{SGELSD}\indexR{CGELSD}
\indexR{SSTEDC}\indexR{CSTEDC}
ISPEC = 10 is used in the driver routines
xSTEVR and xSYEVR/xHEEVR to check for IEEE-754 compliance
in the parameters NaN and Infinity;  if so, these driver routines call xSTEMR;
otherwise, a slower algorithm is selected. 
\indexR{SSTEVR}
\indexR{SSYEVR}\indexR{CHEEVR}
\indexR{SSTEMR}\indexR{CSTEMR}
ISPEC = 12 to 16 is used\index{tuning!block multishift QR}  in the 
block multishift $QR$ algorithm, xHSEQR, and its auxiliaries
\indexR{SHSEQR}\indexR{CHSEQR}

The LAPACK test programs use a special version of ILAENV
\indexR{ILAENV}
where the parameters are set via a COMMON block interface.
This is convenient for experimenting with different values of, say,
the block size in order to exercise different parts of the code
and to compare the relative performance of different parameter values.

\begin{table}[tbh]\centering
\newcommand{\x}{$\bullet$}
\newcommand{\1}{{\small\raisebox{1ex}{\dag}}}
\begin{tabular}{|l|l|c|c|c|} \hline
real & complex & NB & NBMIN & NX \\ \hline\hline
SGBTRF\indexR{SGBTRF} & CGBTRF\indexR{CGBTRF} & \x & & \\
SGEBRD\indexR{SGEBRD} & CGEBRD\indexR{CGEBRD} & \x & \x & \x \\ 
SGEHRD\indexR{SGEHRD} & CGEHRD\indexR{CGEHRD} & \x & \x & \x \\
SGELQF\indexR{SGELQF} & CGELQF\indexR{CGELQF} & \x & \x & \x \\
SGEQLF\indexR{SGEQLF} & CGEQLF\indexR{CGEQLF} & \x & \x & \x \\
SGEQP3\indexR{SGEQP3} & CGEQP3\indexR{CGEQP3} & \x & \x & \x \\
SGEQRF\indexR{SGEQRF} & CGEQRF\indexR{CGEQRF} & \x & \x & \x \\
SGERQF\indexR{SGERQF} & CGERQF\indexR{CGERQF} & \x & \x & \x \\
SGETRF\indexR{SGETRF} & CGETRF\indexR{CGETRF} & \x & & \\
SGETRI\indexR{SGETRI} & CGETRI\indexR{CGETRI} & \x & \x & \\
SLAUUM\indexR{SLAUUM}&CLAUUM\indexR{CLAUUM} & \x & & \\
SORGLQ\indexR{SORGLQ} & CUNGLQ\indexR{CUNGLQ} & \x & \x & \x \\
SORGQL\indexR{SORGQL} & CUNGQL\indexR{CUNGQL} & \x & \x & \x \\
SORGQR\indexR{SORGQR} & CUNGQR\indexR{CUNGQR} & \x & \x & \x \\
SORGRQ\indexR{SORGRQ} & CUNGRQ\indexR{CUNGRQ} & \x & \x & \x \\
SORMLQ\indexR{SORMLQ} & CUNMLQ\indexR{CUNMLQ} & \x & \x & \\
SORMQL\indexR{SORMQL} & CUNMQL\indexR{CUNMQL} & \x & \x & \\
SORMQR\indexR{SORMQR} & CUNMQR\indexR{CUNMQR} & \x & \x & \\
SORMRQ\indexR{SORMRQ} & CUNMRQ\indexR{CUNMRQ} & \x & \x & \\
SORMRZ\indexR{SORMRZ} & CUNMRZ\indexR{CUNMRZ} & \x & \x & \\
SPBTRF\indexR{SPBTRF} & CPBTRF\indexR{CPBTRF} & \x & & \\
SPOTRF\indexR{SPOTRF} & CPOTRF\indexR{CPOTRF} & \x & & \\
%SPOTRI\indexR{SPOTRI} & CPOTRI\indexR{CPOTRI} & \x & & \\
SSTEBZ\indexR{SSTEBZ} &        & \x &    &    \\
SSYGST\indexR{SSYGST} & CHEGST\indexR{CHEGST} & \x & & \\
SSYTRD\indexR{SSYTRD} & CHETRD\indexR{CHETRD} & \x & \x & \x \\
SSYTRF\indexR{SSYTRF} & CHETRF\indexR{CHETRF} & \x & \x & \\
       & CSYTRF\indexR{CSYTRF} & \x & \x & \\
SSYTRF\_ROOK\indexR{SSYTRF\_ROOK}
       & CHETRF\_ROOK\indexR{CHETRF\_ROOK} & \x & \x & \\
       & CSYTRF\_ROOK\indexR{CSYTRF\_ROOK} & \x & \x & \\
STGSYL\indexR{STGSYL} & CTGSYL\indexR{CTGSYL} &  & \x & \\
STRTRI\indexR{STRTRI} & CTRTRI\indexR{CTRTRI} & \x & & \\
STZRQF\indexR{STZRQF} & CTZRQF\indexR{CTZRQF} & \x & & \\ \hline
\end{tabular}
\caption{Use of the block parameters NB, NBMIN, and NX in LAPACK}
\label{nbnx.tab}
\end{table}

