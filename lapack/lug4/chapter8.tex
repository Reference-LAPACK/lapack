\chapter{The LAPACKE C Interface}\label{chapcinterface}

\section{Introduction}

This chapter describes a two-level C interface to LAPACK, consisting of a high-level interface and
a middle-level interface\index{C interface!level, high-level}\index{C interface!level, middle-level}. 
The high-level interface handles all workspace memory allocation
internally, while the middle-level interface requires the user to provide workspace arrays as in the
original Fortran interface. Both interfaces provide support for both column-major and row-major
matrices. Both interfaces, associated macros and type definitions are
contained in the header file {\tt lapacke.h}.

\subsection{Naming Schemes}

The naming scheme for the high-level interface is to take the Fortran LAPACK routine name, make
it lower case, and add the prefix {\tt LAPACKE\_}. For example, the LAPACK subroutine SGETRF
becomes {\tt LAPACKE\_sgetrf}\index{C interface!naming scheme}.

The naming scheme for the middle-level interface is to take the Fortran LAPACK routine name,
make it lower case, then add the prefix {\tt LAPACKE\_} and the suffix {\tt \_work}.
For example, the LAPACK subroutine SGETRF becomes {\tt LAPACKE\_sgetrf\_work}.

\subsection{Complex Types}

Complex data types are defined by the macros {\tt lapack\_complex\_float} and
{\tt lapack\_complex\_double}, which represent single precision and double precision complex data
types respectively\index{C interface!complex types}. 
It is assumed throughout that the real and imaginary components are stored
contiguously in memory, with the real component first. The {\tt lapack\_complex\_float} and
{\tt lapack\_complex\_double} macros can be either C99 Complex types, a C struct defined type, C++ STL complex types, or a custom complex type. See {\tt lapacke.h} for more details.

\subsection{Array Arguments}

Arrays are passed as pointers, not as a pointer to pointers\index{C interface!array arguments}. 
All the LAPACKE routines that take one or more 2D arrays as a pointer receive a single extra parameter of type {\tt int}. This argument must be equal to either {\tt LAPACK\_ROW\_MAJOR} or {\tt LAPACK\_COL\_MAJOR} which are defined in{\tt  lapacke.h}, specifying whether the arrays are stored in row-major or column-major order. If a routine has multiple array inputs, they must all use the same ordering.

Note that using row-major ordering may require more memory and time than column-major ordering, because the routine must transpose the row-major order to the column-major order required by the underlying LAPACK routine.

Each 2D array argument in a Fortran LAPACK routine has an additional argument that specifies its leading dimension. For row-major 2D arrays, elements within a row are assumed to be contiguous and elements from one row to the next are assumed to be a leading dimension apart. For column-major 2D arrays, elements within a column are assumed to be contiguous and elements from one column to the next are assumed to be a leading dimension apart.

\subsection{Aliasing of Arguments}

Unless specified otherwise, only input arguments (that is, scalars passed by values and arrays specified with the const qualifier) may be legally aliased on a call to the C interface to LAPACK
\index{C interface!aliasing of arguments}.

\subsection{INFO Parameters}

The LAPACKE interface functions set their {\tt lapack\_int} return value to the value of the INFO parameter, which contains information such as error and exit conditions. This differs from LAPACK routines, which return this information as a Fortran integer parameter
\index{C interface!INFO parameters}.

In LAPACKE, INFO is used exactly as it is in LAPACK. If INFO returns the row or column number of a matrix using 1-based indexing in Fortran, the value is {\bf not} adjusted for zero-based indexing.

\subsection{NaN Checking}

The high-level interface includes an optional, on by default, NaN check on all matrix inputs before calling any LAPACK routine\index{C interface!NaN checking}. 
This option affects all routines. If the inputs contain any NaNs, the input parameter corresponding matrix will be flagged with an INFO parameter error. For example, if the fifth parameter is found to contain a NaN, the function will return with the value -5.

The NaN check, as well as other parameters, can be disabled by defining {\tt LAPACK\_DISABLE\_NAN\_CHECK} macro in {\tt lapacke.h}. The middle-level interface does not contain the NaN check.

\subsection{Integers}

Variables with the Fortran type integer are converted to {\tt lapack\_int} in LAPACKE
\index{C interface!integers}. This conforms with modifiable integer type size, especially given ILP64 programming model: re-defining {\tt lapack\_int} as {\tt long int} (8 bytes) will be enough to support this model, as {\tt lapack\_int} 
is defined as {\tt int} (4 bytes) by default, supporting LP64 programming model.

\subsection{Logicals}

Fortran logicals are converted to {\tt lapack\_logical}, which is defined as {\tt lapack\_int}.

\subsection{Memory Management}

All memory management is handled by the functions {\tt LAPACKE\_malloc} and {\tt LAPACKE\_free}\index{C interface!memory management}. This allows users to easily use their own memory manager instead of the default by modifying their definitions in {\tt lapacke.h}.

This interface should be thread-safe to the extent that these memory management routines and the underlying LAPACK routines are thread-safe.

\subsection{New Error Codes}

Since the high-level interface does not use work arrays, error notification is needed in the event of a user running out of memory.

If a work array cannot be allocated, {\tt LAPACK\_WORK\_MEMORY\_ERROR} is returned by the function; if there was insufficient memory to complete a transposition,
{\tt LAPACK\_TRANSPOSE\_MEMORY\_ERROR} is returned
\index{C interface!new error codes}.

\section{Function List}

This section lists the currently available LAPACK subroutines that are available in the LAPACKE C interface. The LAPACK base names are given below; the corresponding LAPACKE function name is {\tt LAPACKE\_xbase} or {\tt LAPACKE\_xbase\_work} where {\tt x} is the type:
{\tt s} or {\tt d} for single or double precision real, {\tt c} or {\tt z} for single or double precision complex, with base representing the base name. Function prototypes are given in the file {\tt lapacke.h}. See the LAPACK documentation for detailed information about the routines and their parameters.

\subsection{Real Functions}

The following LAPACK subroutine base names are supported for single precision ({\tt s}) and double precision ({\tt d}), in both the high-level and middle-level interfaces
\index{C interface!real functions}:

\begin{tabbing}
xxxxxx \= 
xxxxxx \= 
xxxxxx \=
xxxxxx \= 
xxxxxx \=
xxxxxx \=
xxxxxx \=
xxxxxx \=
xxxxxx \=
xxxxxx \=
xxxxxx \=
xxxxxx \kill
bdsdc \> 
bdsqr \> 
disna \> 
gbbrd \> 
gbcon \>
gbequ \> 
gbequb \> 
gbrfs \> 
gbrfsx \>
gbsv \>
gbsvx \> 
gbsvxx \\
gbtrf \>
gbtrs \> 
gebak \> 
gebal \> 
gebrd \> 
gecon \> 
geequ \> 
geequb \> 
gees \> 
geesx \> 
geev \> 
geevx \\
gehrd \> 
gejsv \>
gelqf \>
gels \>
gelsd \>
gelss \>
gelsy \> 
geqlf \>
geqp3 \>
geqpf \>
geqrf \>
geqrfp \\
gerfs \>
gerfsx \>
gerqf \>
gesdd \>
gesv \>
gesvd \>
gesvj \>
gesvx \>
gesvxx \>
getrf \>
getri \>
getrs \\
ggbak \>
ggbal \> 
gges \>
ggesx \>
ggev \>
ggevx \>
ggglm \>
gghrd \>
gglse \>
ggqrf \>
ggrqf \>
ggsvd \\
ggsvp \>
gtcon \>
gtrfs \>
gtsv \>
gtsvx \>
gttrf \>
gttrs \>
hgeqz \>
hsein \>
hseqr \>
opgtr \>
opmtr \\
orgbr \>
orghr \>
orglq \>
orgql \>
orgqr \>
orgrq \>
orgtr \>
ormbr \>
ormhr \>
ormlq \>
ormql \>
ormqr \\
ormrq \>
ormrz \>
ormtr \>
pbcon \>
pbequ \>
pbrfs \>
pbstf \>
pbsv \>
pbsvx \>
pbtrf \>
pbtrs \>
pftrf \\
pftri \>
pftrs \>
pocon \>
poequ \>
poequb \>
porfs \>
porfsx \>
posv \>
posvx \>
posvxx \>
potrf \>
potri \\
potrs \>
ppcon \>
ppequ \>
pprfs \>
ppsv \>
ppsvx \>
pptrf \>
pptri \>
pptrs \>
pstrf \>
ptcon \>
pteqr \\
ptrfs \>
ptsv \>
ptsvx \>
pttrf \>
pttrs \>
sbev \>
sbevd \>
sbevx \>
sbgst \>
sbgv \>
sbgvd \>
sbgvx \\
sbtrd \>
sfrk \>
spcon \>
spev \>
spevd \>
spevx \>
spgst \>
spgv \>
spgvd \>
spgvx \>
sprfs \>
spsv \\
spsvx \>
sptrd \>
sptrf \>
sptri \>
sptrs \>
stebz \>
stedc \>
stegr \>
stein \>
stemr \>
steqr \>
sterf \\
stev \>
stevd \>
stevr \>
stevx \>
sycon \>
syequb \>
syev \>
syevd \>
syevr \>
syevx \>
sygst \>
sygv \\
sygvd \>
sygvx \>
syrfs \>
syrfsx \>
sysv \>
sysvx \>
sysvxx \>
sytrd \>
sytrf \>
sytri \>
sytrs \>
tbcon \\
tbrfs \>
tbtrs \>
tfsm \>
tftri \> 
tfttp \>
tfttr \>
tgevc \>
tgexc \>
tgsen \>
tgsja \>
tgsna \>
tgsyl \\
tpcon \>
tprfs \>
tptri \>
tptrs \>
tpttf \>
tpttr \>
trcon \>
trevc \>
trexc \>
trrfs \>
trsen \>
trsna \\
trsyl \>
trtri \> 
trtrs \>
trttf \>
trttp \>
tzrzf \\
\end{tabbing}

\subsection{Complex Functions}

The following LAPACK subroutine base names are supported for complex single precision ({\tt c}) and complex double precision ({\tt z}), in both the high-level and middle-level interfaces
\index{C interface!complex functions}:

\begin{tabbing}
xxxxxx \= 
xxxxxx \= 
xxxxxx \=
xxxxxx \= 
xxxxxx \=
xxxxxx \=
xxxxxx \=
xxxxxx \=
xxxxxx \=
xxxxxx \=
xxxxxx \=
xxxxxx \kill
bdsqr \> 
gbbrd \> 
gbcon \>
gbequ \> 
gbequb \> 
gbrfs \> 
gbrfsx \>
gbsv \>
gbsvx \> 
gbsvxx \>
gbtrf \>
gbtrs \\ 
gebak \> 
gebal \> 
gebrd \> 
gecon \> 
geequ \> 
geequb \> 
gees \> 
geesx \> 
geev \> 
geevx \>
gehrd \> 
gelqf \\
gels \>
gelsd \>
gelss \>
gelsy \> 
geqlf \>
geqp3 \>
geqpf \>
geqrf \>
geqrfp \>
gerfs \>
gerfsx \>
gerqf \\
gesdd \>
gesv \>
gesvd \>
gesvx \>
gesvxx \>
getrf \>
getri \>
getrs \>
ggbak \>
ggbal \> 
gges \\
ggesx \>
ggev \>
ggevx \>
ggglm \>
gghrd \>
gglse \>
ggqrf \>
ggrqf \>
ggsvd \>
ggsvp \>
gtcon \>
gtrfs \\
gtsv \>
gtsvx \>
gttrf \>
gttrs \>
hbev \>
hbevd \>
hbevx \>
hbgst \>
hbgv \>
hbgvd \>
hbgvx \>
hbtrd \\
hecon \>
heequb \>
heev \>
heevd \>
heevr \>
heevx \>
hegst \>
hegv \>
hegvd \>
hegvx \>
herfs \>
herfsx \\
hesv \>
hesvx \>
hesvxx \>
hetrd \>
hetrf \>
hetri \>
hetrs \>
hfrk \>
hgeqz \>
hpcon \>
hpev \>
hpevd \\
hpevx \>
hpgst \>
hpgv \>
hpgvd \>
hpgvx \>
hprfs \>
hpsv \>
hpsvx \>
hptrd \>
hptrf \>
hptri \>
hptrs \\
hsein \>
hseqr \>
pbcon \>
pbequ \>
pbrfs \>
pbstf \>
pbsv \>
pbsvx \>
pbtrf \>
pbtrs \>
pftrf \>
pftri \\
pftrs \>
pocon \>
poequ \>
poequb \>
porfs \>
porfsx \>
posv \>
posvx \>
posvxx \>
potrf \>
potri \>
potrs \\
ppcon \>
ppequ \>
pprfs \>
ppsv \>
ppsvx \>
pptrf \>
pptri \>
pptrs \>
pstrf \>
ptcon \>
pteqr \>
ptrfs \\
ptsv \>
ptsvx \>
pttrf \>
pttrs \>
spcon \>
sprfs \>
spsv \>
spsvx \>
sptrf \>
sptri \>
sptrs \>
stedc \\
stegr \>
stein \>
stemr \>
steqr \>
sycon \>
syequb \>
syrfs \>
syrfsx \>
sysv \>
sysvx \>
sysvxx \>
sytrf \\
sytri \>
sytrs \>
tbcon \>
tbrfs \>
tbtrs \>
tfsm \>
tftri \> 
tfttp \>
tfttr \>
tgevc \>
tgexc \>
tgsen \\
tgsja \>
tgsna \>
tgsyl \>
tpcon \>
tprfs \>
tptri \>
tptrs \>
tpttf \>
tpttr \>
trcon \>
trevc \>
trexc \\
trrfs \>
trsen \>
trsna \>
trsyl \>
trtri \> 
trtrs \>
trttf \>
trttp \>
tzrzf \>
ungbr \>
unghr \>
unglq \\
ungql \>
ungqr \>
ungrq \>
ungtr \>
unmbr \>
unmhr \>
unmlq \>
unmql \>
unmqr \>
unmrq \>
unmrz \>
unmtr \\
upgtr \>
upmtr \\
\end{tabbing}

\subsection{Mixed Precision Functions}

The following LAPACK subroutine base names are supported only for double precision ({\tt ds}) and complex double precision ({\tt zc})\index{C interface!mixed precision functions}:

\begin{tabbing}
xxxxxx \=
xxxxxx \kill
gesv \>
posv \\
\end{tabbing}


\section{Examples}

This section contains examples of calling LAPACKE functions from a C program
\index{C interface!examples}.

\subsection{Calling SGEQRF}\index{C interface examples!calling SGEQRF}

Suppose you wish to call the function SGEQRF, which computes the QR factorization of a single precision real rectangular matrix in LAPACK, and you wish to have the LAPACKE interface handle the necessary work space memory allocation for you.

The base name for this function is {\tt geqrf}, which is included in the list of real functions above. The LAPACKE function name is then constructed by prepending {\tt LAPACK\_} followed by {\tt s} to the base name: {\tt LAPACKE\_sgeqrf}.

We will assume that our matrix is stored in column-major order in the {\tt m}-by-{\tt n} 
array {\tt a}, which has a leading dimension of {\tt lda}. The variable declarations should be as follows:

\begin{verbatim}
    lapack_int m, n, lda, info;
    float *a, *tau;
\end{verbatim}

The LAPACKE function call is then:

\begin{verbatim}
info = LAPACKE_sgeqrf( LAPACK_COL_MAJOR, m, n, a, lda, tau );
\end{verbatim}

\subsection{Calling CUNGQR}\index{C interface examples!calling CUNGQR}

Suppose you wish to call the function CUNGQR, which generates $Q$ from the results of a QR factorization of a single precision complex rectangular matrix, and you wish to provide the required workspace.

The base name for this function is {\tt ungqr}, 
which is included in the list of complex functions above. The LAPACKE function name is then constructed by prepending {\tt LAPACK\_} followed by {\tt c} to the base name, 
with the suffix {\tt \_work} to indicate that the user will supply the work space: 
{\tt LAPACKE\_cungqr\_work}.

We will assume again that our matrix is stored in column-major order in the {\tt m}-by-{\tt n} array {\tt a}, which has a leading dimension of {\tt lda}. 
From the LAPACK documentation, the work space array work must have a length of at least {\tt n}; the length of {\tt work} is given in {\tt lwork}. 
The variable declarations should be as follows:

\begin{verbatim}
   lapack_int m, n, k, lda, lwork, info;
   lapack_complex_float *a, *tau, *work;
\end{verbatim}

The LAPACKE function call is then:

\begin{verbatim}
info = LAPACKE_cungqr_work( LAPACK_COL_MAJOR, m, n, k, a, lda, tau, work, lwork );
\end{verbatim}

\subsection{Calling SGELS}\index{C interface examples!calling SGELS}

In this example, we wish solve the least squares problem $\min_{x} \| B - Ax \|$ for two 
right-hand sides using the LAPACK routine SGELS. For input we will use the 5-by-3 matrix

\[
A = \left({\begin{array}{rrr} 
1 & 1 & 1 \\ 2 & 3 & 4 \\ 3 & 5 & 2 \\ 4 & 2 & 5 \\ 5 & 4 & 3
\end{array} } \right)
\]

and the 5-by-2 matrix

\[
B = \left({\begin{array}{rr} 
 -10 & -3 \\ 12 & 14 \\ 14 & 12 \\ 16 & 16 \\ 18 & 16
\end{array} } \right)
\]

We will first store the input matrix as a static C two-dimensional array, which is stored in row-major order, and let LAPACKE handle the work space array allocation. The LAPACK base name for this function is {\tt gels}, and we will use single precision ({\tt s}), so the LAPACKE function name is {\tt LAPACKE\_sgels}, 
thus {\tt lda} = 3 and {\tt ldb} = 2. The output for each right hand side is stored in {\tt b} as consecutive vectors of length 3. The correct answer for this problem is the 3-by-2 matrix

\[
\left({\begin{array}{rr}  
2 & 1 \\ 1 & 1 \\ 1 & 2
\end{array} } \right)
\]

A complete C program for this example is given below. Note that when the arrays are passed to the LAPACK routine, they must be dereferenced, since LAPACK is expecting arrays of type 
{\tt float~*}, not {\tt float~**}.

\begin{verbatim}
/* Calling SGELS using row-major order */

#include <stdio.h>
#include <lapacke.h>

int main (int argc, const char * argv[])
{
   float a[5][3] = {1,1,1,2,3,4,3,5,2,4,2,5,5,4,3};
   float b[5][2] = {-10,-3,12,14,14,12,16,16,18,16};
   lapack_int info,m,n,lda,ldb,nrhs;
   int i,j;

   m = 5;
   n = 3;
   nrhs = 2;
   lda = 3;
   ldb = 2;

   info = LAPACKE_sgels(LAPACK_ROW_MAJOR,'N',m,n,nrhs,*a,lda,*b,ldb);

   for(i=0;i<n;i++)
   {
      for(j=0;j<nrhs;j++)
      {
         printf("%lf ",b[i][j]);
      }
      printf("\n");
   }
   return(info);
}
\end{verbatim}

Alternatively, we can use column-major ordering for the matrices in this example, as shown below. Here, the matrices are stored as static one-dimensional C arrays. These arrays have a leading dimension that is equal to the number of rows.

\begin{verbatim}
/* Calling SGELS using column-major order */

#include <stdio.h>
#include <lapacke.h>

int main (int argc, const char * argv[])
{
   float a[5*3] = {1,2,3,4,5,1,3,5,2,4,1,4,2,5,3};
   float b[5*2] = {-10,12,14,16,18,-3,14,12,16,16};
   lapack_int info,m,n,lda,ldb,nrhs;
   int i,j;

   m = 5;
   n = 3;
   nrhs = 2;
   lda = 5;
   ldb = 5;

   info = LAPACKE_sgels(LAPACK_COL_MAJOR,'N',m,n,nrhs,a,lda,b,ldb);

   for(i=0;i<n;i++)
   {
      for(j=0;j<nrhs;j++)
      {
         printf("%lf ",b[i+ldb*j]);
      }
      printf("\n");
   }
   return(info);
}
\end{verbatim}

\subsection{Calling CGEQRF and the CBLAS}
\index{C interface examples!calling CGEQRF and CBLAS}

In this example, we will call the LAPACK routine CGEQRF to compute the QR factorization of a matrix. We then call CUNGQR to construct the $Q$ matrix and then use the CBLAS routine CGEMM to compute $Q^{H} Q - I$ to check that $Q$ is Hermitian. The error $\| Q^{H} Q - I \|$ is printed at the end of the program.

In the first version, given below, we let LAPACKE handle the memory allocation for the workspace internally:

\begin{verbatim}
/* Calling CGEQRF and CUNGQR to compute Q without workspace querying */

#include <stdio.h>
#include <stdlib.h>
#include <lapacke.h>
#include <cblas.h>

int main (int argc, const char * argv[])
{
   lapack_complex_float *a,*tau,*r,one,zero;
   lapack_int info,m,n,lda;
   int i,j;
   float err=0.0;
   m = 10;   n = 5;   lda = m;
   one = lapack_make_complex_float(1.0,0.0);
   zero= lapack_make_complex_float(0.0,0.0);
   a = calloc(m*n,sizeof(lapack_complex_float));
   r = calloc(n*n,sizeof(lapack_complex_float));
   tau = calloc(m,sizeof(lapack_complex_float));
   for(j=0;j<n;j++)
      for(i=0;i<m;i++)
         a[i+j*m] = lapack_make_complex_float(i+1,j+1);
   info = LAPACKE_cgeqrf(LAPACK_COL_MAJOR,m,n,a,lda,tau);
   info = LAPACKE_cungqr(LAPACK_COL_MAJOR,m,n,n,a,lda,tau);
   for(j=0;j<n;j++)
      for(i=0;i<n;i++)
         r[i+j*n]=(i==j)?-one:zero;
   cblas_cgemm(CblasColMajor,CblasConjTrans,CblasNoTrans,
               n,n,m,&one,a,lda,a,lda,&one,r,n );
   for(i=0;i<n;i++)
      for(j=0;j<n;j++)
         err=MAX(err,cabs(r[i+j*n]));
   printf("error=%e\n",err);
   free(tau);
   free(r);
   free(a);
   return(info);
}
\end{verbatim}

The second version uses the workspace query facility for both CGEQRF and CUNGQR to obtain the optimal size for the parameter {\tt lwork}, 
which we use to allocate our own workspace in the array {\tt work}:

\begin{verbatim}
/*  Calling CGEQRF and CUNGQR to compute Q with workspace querying */

#include <stdio.h>
#include <stdlib.h>
#include <lapacke.h>
#include <cblas.h>

int main (int argc, const char * argv[])
{
   lapack_complex_float *a,*tau,*r,*work,one,zero,query;
   lapack_int info,m,n,lda,lwork;
   int i,j;
   float err;
   m = 10;   n = 5;   lda = m;
   one = lapack_make_complex_float(1.0,0.0);
   zero= lapack_make_complex_float(0.0,0.0);
   a = calloc(m*n,sizeof(lapack_complex_float));
   r = calloc(n*n,sizeof(lapack_complex_float));
   tau = calloc(m,sizeof(lapack_complex_float));
   for(j=0;j<n;j++)
      for(i=0;i<m;i++)
         a[i+j*m] = lapack_make_complex_float(i+1,j+1);
   info = LAPACKE_cgeqrf_work(LAPACK_COL_MAJOR,m,n,a,lda,tau,&query,-1);
   lwork = (lapack_int)query;
   info = LAPACKE_cungqr_work(LAPACK_COL_MAJOR,m,n,n,a,lda,tau,&query,-1);
   lwork = MAX(lwork,(lapack_int)query);
   work = calloc(lwork,sizeof(lapack_complex_float));
   info = LAPACKE_cgeqrf_work(LAPACK_COL_MAJOR,m,n,a,lda,tau,work,lwork);
   info = LAPACKE_cungqr_work(LAPACK_COL_MAJOR,m,n,n,a,lda,tau,work,lwork);
   for(j=0;j<n;j++)
      for(i=0;i<n;i++)
         r[i+j*n]=(i==j)?-one:zero;
   cblas_cgemm(CblasColMajor,CblasConjTrans,CblasNoTrans,
               n,n,m,&one,a,lda,a,lda,&one,r,n);
   err=0.0;
   for(i=0;i<n;i++)
      for(j=0;j<n;j++)
         err=MAX(err,cabs(r[i+j*n]));
   printf("error=%e\n",err);
   free(work);
   free(tau);
   free(r);
   free(a);
   return(info);
}
}
\end{verbatim}
